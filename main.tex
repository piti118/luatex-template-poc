\documentclass{article}
\usepackage[utf8]{inputenc}
\usepackage{fontspec}
\usepackage{luacode}
\usepackage{babel}
\usepackage{graphicx}
\babelprovide[main,import]{thai}

% Set main font to a Thai-compatible font (requires XeLaTeX or LuaLaTeX)
\setmainfont{TH Sarabun New}

\begin{document}

สวัสดีครับ นี่คือข้อความภาษาไทยในเอกสาร LaTeX


% Lua code to read JSON, use etlua template, and output a table
\begin{luacode*}
local t = require "tgtex.lua"
local json = t.readJSON("example.json")
local template_content = t.readFile("template.tex")

for i, entry in ipairs(json.data) do
  local output = t.renderTemplate(template_content, entry)
  t.printTEX(output)
end
\end{luacode*}

\textbf{กรุงเทพมหานคร} เป็นเมืองหลวงและนครที่มีประชากรมากที่สุดของประเทศไทย เป็นศูนย์กลางการปกครอง การศึกษา การคมนาคมขนส่ง การเงินการธนาคาร การพาณิชย์ การสื่อสาร และความเจริญของประเทศ เป็นเมืองที่มีชื่อยาวที่สุดในโลก ตั้งอยู่บนสามเหลี่ยมปากแม่น้ำเจ้าพระยา มีแม่น้ำเจ้าพระยาไหลผ่านและแบ่งเมืองออกเป็น 2 ฝั่ง คือ ฝั่งพระนครและฝั่งธนบุรี กรุงเทพมหานครมีพื้นที่ทั้งหมด 1,568.737 ตร.กม. มีประชากรตามทะเบียนราษฎรกว่า 5 ล้านคน ทำให้กรุงเทพมหานครเป็นเอกนคร (Primate City) จัด มีผู้กล่าวว่า กรุงเทพมหานครเป็น "เอกนครที่สุดในโลก" เพราะมีประชากรมากกว่านครที่มีประชากรมากเป็นอันดับ 2 ถึง 40 เท่า[3]

มหาวิทยาลัยลัฟเบอระ (Loughborough University) จัดกรุงเทพมหานครว่าเป็นนครโลกระดับแอลฟาลบ[4] กรุงเทพมหานครยังเป็นเมืองที่มีตึกระฟ้ามากที่สุดเป็นอันดับที่ 7 ของโลก[5] มีสถานที่ท่องเที่ยวหลายแห่ง เช่น พระบรมมหาราชวัง พระที่นั่งวิมานเมฆ วัดต่าง ๆ นอกจากนี้ยังมีแหล่งจับจ่ายใช้สอยและค้าขายที่สำคัญซึ่งดึงดูดนักท่องเที่ยวต่างชาติมากมาย โดยในปี พ.ศ. 2555 องค์กรการท่องเที่ยวโลก (UNWTO) ได้จัดอันดับกรุงเทพมหานครเป็นเมืองที่มีคนเดินทางเข้าเป็นอันดับที่ 10 ของโลกและเป็นอันดับที่ 2 ของเอเชีย โดยมีคนเดินทางมากกว่า 26.5 ล้านคน[6] นอกจากนี้จากการจัดอันดับการใช้จ่ายผ่านบัตรเครดิตมาสเตอร์การ์ด ประจำปี พ.ศ. 2557 กรุงเทพมหานครมีการใช้จ่ายผ่านบัตรเครดิตของนักท่องเที่ยวถึง 16.42 ล้านดอลลาร์ เป็นอันดับที่ 2 ของโลก รองจากกรุงลอนดอน สหราชอาณาจักร เท่านั้น[7]

กรุงเทพมหานครเป็นเขตปกครองพิเศษของประเทศไทย มิได้มีสถานะเป็นจังหวัด คำว่า "กรุงเทพมหานคร" นั้นยังใช้เรียกองค์กรปกครองส่วนท้องถิ่นของกรุงเทพมหานครอีกด้วย กรุงเทพมหานครมีการเลือกตั้งผู้บริหารท้องถิ่นโดยตรง แต่ปัจจุบันผู้บริหารกรุงเทพมหานครมาจากการแต่งตั้ง

ในสมัยกรุงศรีอยุธยา กรุงเทพมหานครยังเป็นเพียงสถานีการค้าขนาดเล็กอยู่ที่ปากแม่น้ำเจ้าพระยา ต่อมามีขนาดเพิ่มขึ้นและเป็นที่ตั้งของเมืองหลวง 2 แห่งคือ กรุงธนบุรี ในปี พ.ศ. 2311 และกรุงรัตนโกสินทร์ใน พ.ศ. 2325 กรุงเทพมหานครเป็นหัวใจของการทำให้ประเทศสยามทันสมัยและเป็นเวทีกลางของการต่อสู้ทางการเมืองของประเทศตลอดคริสต์ศตวรรษที่ 20 นครเติบโตอย่างรวดเร็วและปัจจุบันมีผลกระทบสำคัญต่อการเมือง เศรษฐกิจ การศึกษา สื่อและสังคมสมัยใหม่ของไทย ในช่วงที่การลงทุนในเอเชียรุ่งเรือง ทำให้บรรษัทข้ามชาติจำนวนมากเข้ามาตั้งสำนักงานใหญ่ภูมิภาคในกรุงเทพมหานคร ทำให้กรุงเทพมหานครเป็นกำลังหลักทางการเงินและธุรกิจในภูมิภาค นอกจากนี้ยังเป็นศูนย์กลางการขนส่งและสาธารณสุขระหว่างประเทศและกำลังเติบโตเป็นศูนย์กลางศิลปะ แฟชัน และการบันเทิงในภูมิภาค อย่างไรก็ดี การเติบโตอย่างรวดเร็วของกรุงเทพมหานครขาดการวางผังเมือง ทำให้ระบบโครงสร้างพื้นฐานไม่เพียงพอ ถนนที่จำกัดและการใช้รถส่วนบุคคลอย่างกว้างขวางส่งผลให้เกิดปัญหาจราจรแออัดเรื้อรัง
\end{document}
